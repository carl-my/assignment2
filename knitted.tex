\documentclass{article}\usepackage[]{graphicx}\usepackage[]{color}
% maxwidth is the original width if it is less than linewidth
% otherwise use linewidth (to make sure the graphics do not exceed the margin)
\makeatletter
\def\maxwidth{ %
  \ifdim\Gin@nat@width>\linewidth
    \linewidth
  \else
    \Gin@nat@width
  \fi
}
\makeatother

\definecolor{fgcolor}{rgb}{0.345, 0.345, 0.345}
\newcommand{\hlnum}[1]{\textcolor[rgb]{0.686,0.059,0.569}{#1}}%
\newcommand{\hlstr}[1]{\textcolor[rgb]{0.192,0.494,0.8}{#1}}%
\newcommand{\hlcom}[1]{\textcolor[rgb]{0.678,0.584,0.686}{\textit{#1}}}%
\newcommand{\hlopt}[1]{\textcolor[rgb]{0,0,0}{#1}}%
\newcommand{\hlstd}[1]{\textcolor[rgb]{0.345,0.345,0.345}{#1}}%
\newcommand{\hlkwa}[1]{\textcolor[rgb]{0.161,0.373,0.58}{\textbf{#1}}}%
\newcommand{\hlkwb}[1]{\textcolor[rgb]{0.69,0.353,0.396}{#1}}%
\newcommand{\hlkwc}[1]{\textcolor[rgb]{0.333,0.667,0.333}{#1}}%
\newcommand{\hlkwd}[1]{\textcolor[rgb]{0.737,0.353,0.396}{\textbf{#1}}}%
\let\hlipl\hlkwb

\usepackage{framed}
\makeatletter
\newenvironment{kframe}{%
 \def\at@end@of@kframe{}%
 \ifinner\ifhmode%
  \def\at@end@of@kframe{\end{minipage}}%
  \begin{minipage}{\columnwidth}%
 \fi\fi%
 \def\FrameCommand##1{\hskip\@totalleftmargin \hskip-\fboxsep
 \colorbox{shadecolor}{##1}\hskip-\fboxsep
     % There is no \\@totalrightmargin, so:
     \hskip-\linewidth \hskip-\@totalleftmargin \hskip\columnwidth}%
 \MakeFramed {\advance\hsize-\width
   \@totalleftmargin\z@ \linewidth\hsize
   \@setminipage}}%
 {\par\unskip\endMakeFramed%
 \at@end@of@kframe}
\makeatother

\definecolor{shadecolor}{rgb}{.97, .97, .97}
\definecolor{messagecolor}{rgb}{0, 0, 0}
\definecolor{warningcolor}{rgb}{1, 0, 1}
\definecolor{errorcolor}{rgb}{1, 0, 0}
\newenvironment{knitrout}{}{} % an empty environment to be redefined in TeX

\usepackage{alltt}
\title{Statistical Programming with R\\Assignment 2}

\author{\small Tove Henning\\ \small Carl Munkby\\ \small Johannes Zetterberg}
\date{}

\usepackage[margin = 1in]{geometry}
\IfFileExists{upquote.sty}{\usepackage{upquote}}{}
\begin{document}
\maketitle

\pagebreak

\section{Least squares variance simulation}
\begin{knitrout}
\definecolor{shadecolor}{rgb}{0.969, 0.969, 0.969}\color{fgcolor}\begin{kframe}
\begin{alltt}
\hlkwd{library}\hlstd{(StatProg)}
\hlkwd{library}\hlstd{(tidyverse)}
\end{alltt}


{\ttfamily\noindent\itshape\color{messagecolor}{\#\# -- Attaching packages ----------------------------------------- tidyverse 1.3.0 --}}

{\ttfamily\noindent\itshape\color{messagecolor}{\#\# v ggplot2 3.3.2\ \ \ \  v purrr\ \  0.3.4\\\#\# v tibble\ \ 3.0.3\ \ \ \  v dplyr\ \  1.0.2\\\#\# v tidyr\ \  1.1.2\ \ \ \  v stringr 1.4.0\\\#\# v readr\ \  1.3.1\ \ \ \  v forcats 0.5.0}}

{\ttfamily\noindent\itshape\color{messagecolor}{\#\# -- Conflicts -------------------------------------------- tidyverse\_conflicts() --\\\#\# x dplyr::filter() masks stats::filter()\\\#\# x dplyr::lag()\ \ \ \ masks stats::lag()}}\begin{alltt}
\hlcom{#### OLS}
\hlstd{olsFun} \hlkwb{<-} \hlkwa{function}\hlstd{(}\hlkwc{data}\hlstd{)\{}
  \hlcom{### set column names and add intercept column to X}
  \hlstd{Y} \hlkwb{<-} \hlstd{data[,}\hlnum{1}\hlstd{]}
  \hlstd{N} \hlkwb{<-} \hlkwd{nrow}\hlstd{(data)}
  \hlstd{X} \hlkwb{<-} \hlkwd{cbind}\hlstd{(}\hlkwd{rep}\hlstd{(}\hlnum{1}\hlstd{, N), data[,}\hlnum{2}\hlstd{])}

  \hlcom{### calculate the formel and extract the Beta coefficient }
  \hlstd{beta_ols}\hlkwb{=}\hlstd{(}\hlkwd{solve}\hlstd{(}\hlkwd{t}\hlstd{(X)}\hlopt \hlstd{X)} \hlopt \hlstd{(}\hlkwd{t}\hlstd{(X)} \hlopt \hlstd{Y))[}\hlnum{2}\hlstd{,}\hlnum{1}\hlstd{]}

  \hlkwd{return}\hlstd{(beta_ols)}
\hlstd{\}}

\hlcom{#Här är en kommentar om OLS. blabla bla blabakljdgklajkl}

\hlstd{testData} \hlkwb{<-} \hlkwd{cbind}\hlstd{(} \hlkwd{c}\hlstd{(}\hlnum{0.62}\hlstd{,} \hlnum{0.18}\hlstd{,} \hlnum{3.92}\hlstd{,} \hlnum{0.80}\hlstd{,} \hlopt{-}\hlnum{5.15}\hlstd{),}
\hlkwd{c}\hlstd{(}\hlnum{0.44}\hlstd{,} \hlnum{1.49}\hlstd{,} \hlnum{0.69}\hlstd{,} \hlnum{0.13}\hlstd{,} \hlnum{1.90}\hlstd{) )}

\hlkwd{olsFun}\hlstd{(}\hlkwc{data} \hlstd{= testData)}
\end{alltt}
\begin{verbatim}
## [1] -3.092464
\end{verbatim}
\end{kframe}
\end{knitrout}

\begin{knitrout}
\definecolor{shadecolor}{rgb}{0.969, 0.969, 0.969}\color{fgcolor}\begin{kframe}
\begin{alltt}
\hlcom{##### weighted least squares}
\hlstd{wlsFun} \hlkwb{<-} \hlkwa{function}\hlstd{(}\hlkwc{data}\hlstd{,} \hlkwc{lambda}\hlstd{)\{}
  \hlkwa{if} \hlstd{(}\hlkwd{is.numeric}\hlstd{(lambda)}\hlopt{==}\hlnum{FALSE}\hlstd{)\{}\hlkwd{print}\hlstd{(}\hlstr{"lambda is not numeric"}\hlstd{) \}}
  \hlkwa{else}\hlstd{\{}
    \hlcom{### create variable for the number of observations in the dataset}
    \hlstd{N} \hlkwb{<-} \hlkwd{nrow}\hlstd{(data)}

    \hlcom{### set column names and add intercept column for X}
    \hlstd{Y} \hlkwb{<-} \hlstd{data[,}\hlnum{1}\hlstd{]}
    \hlstd{X} \hlkwb{<-} \hlkwd{cbind}\hlstd{(}\hlkwd{rep}\hlstd{(}\hlnum{1}\hlstd{,N), data[,}\hlnum{2}\hlstd{])}

    \hlcom{### create a zero matrix N x N}
    \hlstd{Z} \hlkwb{<-} \hlkwd{matrix}\hlstd{(}\hlnum{0}\hlstd{, N, N)}

    \hlcom{## make a forloop to put in the error terms on the diagonal }
    \hlcom{## to create the error covariance matrix}
    \hlstd{er} \hlkwb{<-} \hlkwa{NULL}
    \hlkwa{for} \hlstd{(i} \hlkwa{in} \hlnum{1}\hlopt{:}\hlstd{N) \{}
      \hlstd{er[i]} \hlkwb{<-} \hlkwd{exp}\hlstd{(X[i,}\hlnum{2}\hlstd{]}\hlopt{*}\hlstd{lambda)}
      \hlstd{Z[i,i]} \hlkwb{<-} \hlstd{er[i]}
    \hlstd{\}}
    \hlcom{### calculate and extract the Beta coefficient}
    \hlstd{beta_wls} \hlkwb{=} \hlstd{((}\hlkwd{solve}\hlstd{(}\hlkwd{t}\hlstd{(X)}\hlopt\hlstd{(}\hlkwd{solve}\hlstd{(Z))}\hlopt\hlstd{X)}\hlopt\hlkwd{t}\hlstd{(X)}\hlopt
                   \hlstd{(}\hlkwd{solve}\hlstd{(Z))}\hlopt\hlstd{Y))[}\hlnum{2}\hlstd{,}\hlnum{1}\hlstd{]}
  \hlkwd{return}\hlstd{(beta_wls)}
  \hlstd{\}}
\hlstd{\}}

\hlkwd{wlsFun}\hlstd{(}\hlkwc{data} \hlstd{= testData,} \hlkwc{lambda} \hlstd{=} \hlnum{2}\hlstd{)}
\end{alltt}
\begin{verbatim}
## [1] 0.007392226
\end{verbatim}
\end{kframe}
\end{knitrout}

\subsubsection*{Feasible weighted least squares}
\begin{knitrout}
\definecolor{shadecolor}{rgb}{0.969, 0.969, 0.969}\color{fgcolor}\begin{kframe}
\begin{alltt}
\hlstd{fwlsFun} \hlkwb{<-} \hlkwa{function}\hlstd{(}\hlkwc{data}\hlstd{,} \hlkwc{trueVar}\hlstd{)\{}
  \hlstd{y} \hlkwb{=} \hlstd{data[,}\hlnum{1}\hlstd{]}                             \hlcom{# y is given the value of the first column in the data  }
  \hlstd{N} \hlkwb{<-} \hlkwd{nrow}\hlstd{(data)}                          \hlcom{# N is given the value of the number of rows in the data}
  \hlstd{X} \hlkwb{=} \hlkwd{cbind}\hlstd{(}\hlkwd{rep}\hlstd{(}\hlnum{1}\hlstd{,N), data[,}\hlnum{2}\hlopt{:}\hlkwd{ncol}\hlstd{(data)])} \hlcom{# X is given the value of the remaining columns in the data }
                                           \hlcom{# and an intercept is added}

  \hlstd{mod} \hlkwb{=} \hlkwd{lm}\hlstd{(y} \hlopt{~ -}\hlnum{1} \hlopt{+}\hlstd{X)}         \hlcom{# We use lm() to estimate a linear regression model, mod, with an intercept }
  \hlstd{res} \hlkwb{=} \hlstd{mod}\hlopt{$}\hlstd{residuals}         \hlcom{# res is assigned the value of the residuals of mod}
  \hlstd{res2} \hlkwb{=} \hlstd{res}\hlopt{^}\hlnum{2}

  \hlstd{ln_res2} \hlkwb{=} \hlkwd{lm}\hlstd{(}\hlkwd{log}\hlstd{(res2)} \hlopt{~  -}\hlnum{1} \hlopt{+}\hlstd{X)}        \hlcom{# We use the lm() function to estimate a new model}
                                          \hlcom{# In this model, the dependent variable is ln(res2), }
                                          \hlcom{# which is explained by X*lambda + v}
  \hlstd{lambda_hat} \hlkwb{=} \hlstd{ln_res2}\hlopt{$}\hlstd{coefficients[}\hlnum{2}\hlstd{]}    \hlcom{# Thereby, the value of lambda_hat is the value of }
                                          \hlcom{# the second estimated coefficient, }
                                          \hlcom{# since the first is the intercept }

  \hlcom{# We check if trueVar is TRUE, if so, we assume the structure of the error variance to be exp(x*lambda)}
  \hlkwa{if}\hlstd{(trueVar} \hlopt{==} \hlnum{TRUE}\hlstd{)\{}
    \hlstd{error_cov} \hlkwb{=} \hlkwd{matrix}\hlstd{(}\hlnum{0}\hlstd{, N, N)}                 \hlcom{# We create an empty square matrix }
    \hlkwa{for}\hlstd{(i} \hlkwa{in} \hlnum{1}\hlopt{:}\hlstd{N)\{}                              \hlcom{# The loop repeats itself for every row of X}
      \hlstd{error_cov[i,i]} \hlkwb{<-} \hlkwd{exp}\hlstd{(X[i,}\hlnum{2}\hlstd{]}\hlopt{*}\hlstd{lambda_hat)}  \hlcom{# We fill the diagonal in the empty square matrix }
                                                \hlcom{# with the error variance, exp(x*lambda_hat) for each value of x}
    \hlstd{\}}

  \hlstd{\}}
  \hlcom{# We check if trueVar is TRUE, if not, we assume the structure of the error variance to be 1 + x*lambda}
  \hlkwa{else if}\hlstd{(trueVar} \hlopt{==} \hlnum{FALSE}\hlstd{)}
  \hlstd{\{}
    \hlstd{error_cov} \hlkwb{=} \hlkwd{matrix}\hlstd{(}\hlnum{0}\hlstd{, N, N)}                 \hlcom{# We create an empty square matrix.}
    \hlkwa{for}\hlstd{(i} \hlkwa{in} \hlnum{1}\hlopt{:}\hlstd{N)\{}                              \hlcom{# The loop repeats itself for every row of X.}
      \hlstd{error_cov[i,i]} \hlkwb{<-} \hlnum{1} \hlopt{+} \hlstd{X[i,}\hlnum{2}\hlstd{]}\hlopt{*}\hlstd{lambda_hat}   \hlcom{# We fill the diagonal in the empty square matrix }
                                                \hlcom{# with the error variance, 1 + x*lambda_hat, for each value of x}
    \hlstd{\}}

  \hlstd{\}}
\hlstd{beta_fwls} \hlkwb{=} \hlstd{(}\hlkwd{solve}\hlstd{(}\hlkwd{t}\hlstd{(X)}\hlopt\hlkwd{solve}\hlstd{(error_cov)}\hlopt\hlstd{X)}\hlopt\hlkwd{t}\hlstd{(X)}\hlopt\hlkwd{solve}\hlstd{(error_cov)}\hlopt\hlstd{y)[}\hlnum{2}\hlstd{,}\hlnum{1}\hlstd{]}
\hlcom{# beta_fwls is calculated and returned }

\hlkwd{return}\hlstd{(beta_fwls)}
\hlstd{\}}
\hlkwd{fwlsFun}\hlstd{(}\hlkwc{data} \hlstd{= testData,} \hlkwc{trueVar} \hlstd{=} \hlnum{TRUE}\hlstd{)}
\end{alltt}
\begin{verbatim}
## [1] -2.615266
\end{verbatim}
\begin{alltt}
\hlkwd{fwlsFun}\hlstd{(}\hlkwc{data} \hlstd{= testData,} \hlkwc{trueVar} \hlstd{=} \hlnum{FALSE}\hlstd{)}
\end{alltt}
\begin{verbatim}
## [1] -2.750474
\end{verbatim}
\end{kframe}
\end{knitrout}

\subsubsection*{Feasible weighted least squares}

To estimate beta for the feasible weighted least squares estimator we needed to assume some structure for the error variance depending on whether the variance form is true ($\sigma^2_{\epsilon_i} = e^{x_i\lambda}$) or false ($\sigma^2_{\epsilon_i} = 1 + x_i\lambda$). The data is divided into y and X, y being the dependent variable, X being the explanatory variable. A linear regression model of y and X is estimated and the residuals from this model are used to estimate another linear regression model. In this model, the dependent variable is the natural logarithm of the squared residuals and the independent variable is X. The coefficient of this linear model is the estimated value of $\lambda$, that is, $\hat{\lambda}$. This value of $\hat{\lambda}$ is then used to get an estimate of the error covariance matrix $\Omega(\hat{\lambda})$, where the diagonal is filled with the value of the estimated error variance for each value of x, $\hat{\sigma}^2_{\epsilon_i}$, which differs depending on whether the variance form is assumed to be true of false. 
The value of the feasible weighted least squares estimator is then computed by:$\hat{\beta}_{FWLS}=(\textbf{X}'\Omega(\hat{\lambda})^{-1}\textbf{X})^{-1}\textbf{X}'\Omega(\hat{\lambda})^{-1}\textbf{y}$.


\begin{knitrout}
\definecolor{shadecolor}{rgb}{0.969, 0.969, 0.969}\color{fgcolor}\begin{kframe}
\begin{alltt}
\hlcom{#### Data simulation}
\hlstd{DataFun} \hlkwb{<-} \hlkwa{function}\hlstd{(}\hlkwc{n}\hlstd{,} \hlkwc{lambda}\hlstd{) \{}

    \hlcom{# independent variable}
    \hlstd{x} \hlkwb{<-} \hlkwd{runif}\hlstd{(n,} \hlkwc{min} \hlstd{=} \hlnum{0}\hlstd{,} \hlkwc{max} \hlstd{=} \hlnum{2}\hlstd{)}

    \hlcom{# standard deviation in epsilons normal distribution}
    \hlstd{s} \hlkwb{<-} \hlkwa{NULL}
    \hlkwa{for} \hlstd{(i} \hlkwa{in} \hlkwd{seq_len}\hlstd{(n)) \{}
        \hlstd{s[i]} \hlkwb{<-} \hlkwd{exp}\hlstd{(x[i]}\hlopt{*}\hlstd{lambda)}
    \hlstd{\}}

    \hlcom{# error term}
    \hlstd{epsilon} \hlkwb{<-} \hlkwd{rnorm}\hlstd{(n,} \hlkwc{mean} \hlstd{=} \hlkwd{rep}\hlstd{(}\hlnum{0}\hlstd{, n),} \hlkwc{sd} \hlstd{= s)}

    \hlstd{beta} \hlkwb{<-} \hlnum{2}

    \hlcom{# dependent variable}
    \hlstd{y} \hlkwb{<-} \hlkwa{NULL}
    \hlkwa{for} \hlstd{(i} \hlkwa{in} \hlkwd{seq_len}\hlstd{(n)) \{}
        \hlstd{y[i]} \hlkwb{<-} \hlstd{beta}\hlopt{*}\hlstd{x[i]} \hlopt{+} \hlstd{epsilon[i]}
    \hlstd{\}}

    \hlcom{# container matrix}
    \hlstd{mat} \hlkwb{<-} \hlkwd{matrix}\hlstd{(}\hlkwc{data} \hlstd{=} \hlnum{0}\hlstd{,} \hlkwc{ncol} \hlstd{=} \hlnum{2}\hlstd{,} \hlkwc{nrow} \hlstd{= n)}

    \hlcom{# creating matrix of indepedent and dependent variables}
    \hlkwa{for} \hlstd{(i} \hlkwa{in} \hlkwd{seq_len}\hlstd{(n)) \{}
        \hlstd{mat[i,} \hlnum{1}\hlstd{]} \hlkwb{<-} \hlstd{y[i]}
        \hlstd{mat[i,} \hlnum{2}\hlstd{]} \hlkwb{<-} \hlstd{x[i]}
    \hlstd{\}}
    \hlkwd{return}\hlstd{(mat)}
\hlstd{\}}
\end{alltt}
\end{kframe}
\end{knitrout}

Data generation...

The DataFun funktion generates random data with an error term dependent on the covariance strukture \lambda and the number of observations chosen. The funktion starts by generating the standard deviation of epsilon since the error term is dependent on x, which comes from a uniform distrubtion. When all the parameters are calculated the funktion calculates y and then all the different values of y and x are saved in a matrix and that matrix is set to be the return value.


\begin{knitrout}
\definecolor{shadecolor}{rgb}{0.969, 0.969, 0.969}\color{fgcolor}\begin{kframe}
\begin{alltt}
\hlstd{SimFun} \hlkwb{<-} \hlkwa{function}\hlstd{(}\hlkwc{n}\hlstd{,} \hlkwc{sim_reps}\hlstd{,} \hlkwc{seed}\hlstd{,} \hlkwc{lambda}\hlstd{) \{}
    \hlkwd{set.seed}\hlstd{(seed)}
    \hlstd{R} \hlkwb{<-} \hlstd{sim_reps}

    \hlcom{# saving betas}
    \hlstd{mat} \hlkwb{<-} \hlkwd{matrix}\hlstd{(}\hlnum{0}\hlstd{,} \hlkwc{nrow} \hlstd{= R,} \hlkwc{ncol} \hlstd{=} \hlnum{4}\hlstd{)}

    \hlkwa{for} \hlstd{(i} \hlkwa{in} \hlkwd{seq_len}\hlstd{(R)) \{}
    \hlcom{# data sim}
    \hlstd{dat} \hlkwb{<-} \hlkwd{DataFun}\hlstd{(n, lambda)}
    \hlcom{# estimate sim}
    \hlstd{mat[i,}\hlnum{1}\hlstd{]} \hlkwb{<-} \hlkwd{olsFun}\hlstd{(dat)}
    \hlstd{mat[i,}\hlnum{2}\hlstd{]} \hlkwb{<-} \hlkwd{wlsFun}\hlstd{(dat, lambda)}
    \hlstd{mat[i,}\hlnum{4}\hlstd{]}\hlkwb{<-} \hlkwd{fwlsFun}\hlstd{(dat,} \hlkwc{trueVar} \hlstd{=} \hlnum{TRUE}\hlstd{)}
    \hlstd{mat[i,}\hlnum{3}\hlstd{]}\hlkwb{<-} \hlkwd{fwlsFun}\hlstd{(dat,} \hlkwc{trueVar} \hlstd{=} \hlnum{FALSE}\hlstd{)}
    \hlstd{\}}
    \hlstd{betas} \hlkwb{<-} \hlkwd{apply}\hlstd{(mat,} \hlnum{2}\hlstd{, var)}

    \hlkwd{return}\hlstd{(betas)}
\hlstd{\}}
\end{alltt}
\end{kframe}
\end{knitrout}

The funktion SimFun uses all the pre-existing funktions to generate data and fit models depending on the number of simulationsone wants to perform. The funktion then returns all the fitted betas from all of the models. Except from number of simulation one can also freely choose the number of observations for the data aswell, the seed number and the value of \lambda. 

\begin{knitrout}
\definecolor{shadecolor}{rgb}{0.969, 0.969, 0.969}\color{fgcolor}\begin{kframe}
\begin{alltt}
\hlcom{##### 4. Plot variance estimates}
\hlstd{x} \hlkwb{<-} \hlkwd{c}\hlstd{(}\hlnum{25}\hlstd{,} \hlnum{50}\hlstd{,} \hlnum{100}\hlstd{,} \hlnum{200}\hlstd{,} \hlnum{400}\hlstd{)}

\hlstd{var_obs} \hlkwb{<-} \hlkwd{matrix}\hlstd{(}\hlnum{0}\hlstd{,} \hlkwc{ncol} \hlstd{=} \hlnum{4}\hlstd{,} \hlkwc{nrow} \hlstd{=} \hlnum{5}\hlstd{)}

\hlkwa{for} \hlstd{(i} \hlkwa{in} \hlkwd{seq_along}\hlstd{(x)) \{}
    \hlstd{var_obs[i,]} \hlkwb{<-} \hlstd{(}\hlkwd{SimFun}\hlstd{(x[i],} \hlnum{100}\hlstd{,} \hlnum{2020}\hlstd{,} \hlnum{2}\hlstd{))}
\hlstd{\}}

\hlstd{var_obs} \hlkwb{=} \hlkwd{as.data.frame}\hlstd{(var_obs)}
\hlkwd{rownames}\hlstd{(var_obs)} \hlkwb{=} \hlkwd{c}\hlstd{(}\hlnum{25}\hlstd{,} \hlnum{50}\hlstd{,} \hlnum{100}\hlstd{,} \hlnum{200}\hlstd{,} \hlnum{400}\hlstd{)}


\hlkwd{colnames}\hlstd{(var_obs)} \hlkwb{=} \hlkwd{c}\hlstd{(}\hlstr{"OLS"}\hlstd{,}\hlstr{"WLS"}\hlstd{,}\hlstr{"FWLST"}\hlstd{,}\hlstr{"FWLSF"}\hlstd{)}
\hlkwd{rownames}\hlstd{(var_obs)} \hlkwb{=} \hlkwd{c}\hlstd{(}\hlstr{"25"}\hlstd{,}\hlstr{"50"}\hlstd{,}\hlstr{"100"}\hlstd{,}\hlstr{"200"}\hlstd{,}\hlstr{"400"}\hlstd{)}

\hlkwd{ggplot}\hlstd{(}\hlkwd{as.data.frame}\hlstd{(var_obs),}\hlkwd{aes}\hlstd{(}\hlkwc{x}\hlstd{=}\hlkwd{as.numeric}\hlstd{(}\hlkwd{rownames}\hlstd{(var_obs))))} \hlopt{+}
  \hlkwd{geom_line}\hlstd{(}\hlkwd{aes}\hlstd{(}\hlkwc{y} \hlstd{= OLS,} \hlkwc{colour} \hlstd{=} \hlstr{"OLS"}\hlstd{))} \hlopt{+}
  \hlkwd{geom_line}\hlstd{(}\hlkwd{aes}\hlstd{(}\hlkwc{y} \hlstd{= WLS,} \hlkwc{colour} \hlstd{=} \hlstr{"WLS"}\hlstd{))} \hlopt{+}
  \hlkwd{geom_line}\hlstd{(}\hlkwd{aes}\hlstd{(}\hlkwc{y} \hlstd{= FWLST,} \hlkwc{colour} \hlstd{=} \hlstr{"FWLST"}\hlstd{))} \hlopt{+}
  \hlkwd{geom_line}\hlstd{(}\hlkwd{aes}\hlstd{(}\hlkwc{y} \hlstd{= FWLSF,} \hlkwc{colour} \hlstd{=} \hlstr{"FWLSF"}\hlstd{))} \hlopt{+}
  \hlkwd{labs}\hlstd{(}\hlkwc{x}\hlstd{=}\hlstr{"Number of obsvervations"}\hlstd{,}\hlkwc{y}\hlstd{=}\hlstr{"Variance"}\hlstd{,}
       \hlkwc{title}\hlstd{=}\hlstr{"Variance estimates of beta for each function vs sample size"}\hlstd{)} \hlopt{+}
  \hlkwd{theme}\hlstd{(}\hlkwc{legend.title} \hlstd{=} \hlkwd{element_blank}\hlstd{())} \hlopt{+}
  \hlkwd{scale_x_continuous}\hlstd{(}\hlkwc{breaks}\hlstd{=}\hlkwd{c}\hlstd{(}\hlnum{25}\hlstd{,}\hlnum{50}\hlstd{,}\hlnum{100}\hlstd{,}\hlnum{200}\hlstd{,}\hlnum{400}\hlstd{),}\hlkwc{limits}\hlstd{=}\hlkwd{c}\hlstd{(}\hlnum{25}\hlstd{,} \hlnum{400}\hlstd{))}
\end{alltt}
\end{kframe}
\includegraphics[width=\maxwidth]{figure/unnamed-chunk-6-1} 
\begin{kframe}\begin{alltt}
\hlcom{#############}
\end{alltt}
\end{kframe}
\end{knitrout}

\section{EM algorithm for mixture of normals}

\begin{knitrout}
\definecolor{shadecolor}{rgb}{0.969, 0.969, 0.969}\color{fgcolor}\begin{kframe}
\begin{alltt}
\hlcom{##################################################}
\hlcom{# start of part2}
\hlcom{##################################################}
\hlstd{galaxies} \hlkwb{<-} \hlkwd{as.data.frame}\hlstd{(galaxies)}
\hlkwd{names}\hlstd{(galaxies)} \hlkwb{<-} \hlstr{"km"}

\hlkwd{ggplot}\hlstd{(galaxies,} \hlkwd{aes}\hlstd{(}\hlkwc{x} \hlstd{= km))} \hlopt{+}
  \hlkwd{geom_density}\hlstd{()}
\end{alltt}
\end{kframe}
\includegraphics[width=\maxwidth]{figure/unnamed-chunk-7-1} 
\begin{kframe}\begin{alltt}
\hlstd{galaxies} \hlkwb{=} \hlstd{galaxies}\hlopt{$}\hlstd{km}

\hlstd{gammaUpdate} \hlkwb{=} \hlkwa{function}\hlstd{(}\hlkwc{x}\hlstd{,} \hlkwc{mu}\hlstd{,} \hlkwc{sigma}\hlstd{,} \hlkwc{pi}\hlstd{)\{}
  \hlstd{pdf} \hlkwb{=} \hlkwd{t}\hlstd{(}\hlkwd{sapply}\hlstd{(x, dnorm,} \hlkwc{mean} \hlstd{= mu,} \hlkwc{sd} \hlstd{= sigma))}
  \hlkwa{for}\hlstd{(n} \hlkwa{in} \hlnum{1}\hlopt{:}\hlkwd{length}\hlstd{(x))\{}
    \hlkwa{for}\hlstd{(k} \hlkwa{in} \hlnum{1}\hlopt{:}\hlkwd{length}\hlstd{(mu))\{}
      \hlstd{pdf[n, k]} \hlkwb{=} \hlstd{pi[k]}\hlopt{*}\hlstd{pdf[n, k]}
    \hlstd{\}}
  \hlstd{\}}
  \hlstd{gamma} \hlkwb{=} \hlkwd{as.data.frame}\hlstd{(}\hlkwd{matrix}\hlstd{(}\hlnum{NA}\hlstd{,} \hlkwc{ncol} \hlstd{=} \hlkwd{length}\hlstd{(pi),} \hlkwc{nrow} \hlstd{=} \hlkwd{length}\hlstd{(x)))}
  \hlkwa{for}\hlstd{(n} \hlkwa{in} \hlnum{1}\hlopt{:}\hlkwd{length}\hlstd{(x))\{}
    \hlkwa{for}\hlstd{(k} \hlkwa{in} \hlnum{1}\hlopt{:}\hlkwd{length}\hlstd{(mu))\{}
      \hlstd{gamma[n, k]} \hlkwb{=} \hlstd{pdf[n, k]}\hlopt{/}\hlkwd{sum}\hlstd{(pdf[n,])}
    \hlstd{\}}
  \hlstd{\}}
  \hlkwd{return}\hlstd{(}\hlkwd{as.data.frame}\hlstd{(gamma))}
\hlstd{\}}

\hlcom{# mu}
\hlstd{muUpdate} \hlkwb{=} \hlkwa{function}\hlstd{(}\hlkwc{x}\hlstd{,} \hlkwc{gamma}\hlstd{)\{}
  \hlstd{K} \hlkwb{<-} \hlkwd{ncol}\hlstd{(gamma)}
  \hlstd{mu} \hlkwb{<-} \hlkwa{NULL}
  \hlkwa{for} \hlstd{(i} \hlkwa{in} \hlkwd{seq_len}\hlstd{(K)) \{}
    \hlstd{mu[i]} \hlkwb{<-} \hlkwd{sum}\hlstd{(gamma[,i]}\hlopt{*}\hlstd{x)}\hlopt{/}\hlkwd{sum}\hlstd{(gamma[,i])}
  \hlstd{\}}
  \hlkwd{return}\hlstd{(mu)}
\hlstd{\}}


\hlcom{### Sigma}
\hlstd{sigmaUpdate} \hlkwb{=} \hlkwa{function}\hlstd{(}\hlkwc{x}\hlstd{,} \hlkwc{gamma}\hlstd{,} \hlkwc{mu}\hlstd{)\{}
  \hlstd{N} \hlkwb{=} \hlkwd{matrix}\hlstd{(}\hlnum{0}\hlstd{,} \hlkwc{ncol}\hlstd{=} \hlkwd{ncol}\hlstd{(gamma))}
  \hlstd{sigma} \hlkwb{=} \hlkwd{matrix}\hlstd{(}\hlnum{0}\hlstd{,} \hlkwc{ncol} \hlstd{=} \hlkwd{ncol}\hlstd{(gamma))}
  \hlkwa{for}\hlstd{(k} \hlkwa{in} \hlnum{1}\hlopt{:}\hlkwd{ncol}\hlstd{(gamma))\{}
    \hlkwa{for}\hlstd{(n} \hlkwa{in} \hlnum{1}\hlopt{:}\hlkwd{length}\hlstd{(x))\{}
      \hlstd{sigma[k]} \hlkwb{=} \hlstd{sigma[k]} \hlopt{+} \hlstd{gamma[n,k]}\hlopt{*}\hlstd{(x[n]}\hlopt{-}\hlstd{mu[k])}\hlopt{^}\hlnum{2}
      \hlstd{N[k]} \hlkwb{=} \hlstd{N[k]} \hlopt{+} \hlstd{gamma[n,k]}
    \hlstd{\}}
    \hlstd{sigma[k]} \hlkwb{=} \hlkwd{sqrt}\hlstd{(sigma[k]}\hlopt{/}\hlstd{N[k])}
  \hlstd{\}}
  \hlkwd{return}\hlstd{(sigma)}
\hlstd{\}}

\hlstd{piUpdate} \hlkwb{=} \hlkwa{function}\hlstd{(}\hlkwc{gamma}\hlstd{)\{}
  \hlstd{pi} \hlkwb{<-} \hlkwa{NULL}
  \hlkwa{for} \hlstd{(i} \hlkwa{in} \hlnum{1}\hlopt{:}\hlkwd{ncol}\hlstd{(gamma)) \{}
    \hlstd{pi[i]} \hlkwb{<-} \hlkwd{sum}\hlstd{(gamma[,i])}\hlopt{/}\hlkwd{sum}\hlstd{(gamma)}
  \hlstd{\}}
  \hlkwd{return}\hlstd{(pi)}
\hlstd{\}}

\hlcom{### Log-likelihood }
\hlstd{loglik} \hlkwb{=} \hlkwa{function}\hlstd{(}\hlkwc{x}\hlstd{,} \hlkwc{pi}\hlstd{,} \hlkwc{mu}\hlstd{,} \hlkwc{sigma}\hlstd{)\{}
  \hlstd{sum_pdf} \hlkwb{=} \hlkwd{matrix}\hlstd{(}\hlnum{0}\hlstd{,} \hlkwc{nrow} \hlstd{=} \hlkwd{length}\hlstd{(x))}
  \hlstd{loglike} \hlkwb{=} \hlnum{0}
  \hlkwa{for}\hlstd{(n} \hlkwa{in} \hlnum{1}\hlopt{:}\hlkwd{length}\hlstd{(x))\{}

    \hlkwa{for}\hlstd{(k} \hlkwa{in} \hlnum{1}\hlopt{:}\hlkwd{length}\hlstd{(pi))\{}
      \hlstd{sum_pdf[n]} \hlkwb{=} \hlstd{sum_pdf[n]} \hlopt{+} \hlstd{(pi[k]} \hlopt{*} \hlkwd{dnorm}\hlstd{(x[n], mu[k], sigma[k]))}
    \hlstd{\}}
    \hlstd{loglike} \hlkwb{=} \hlstd{loglike} \hlopt{+} \hlkwd{log}\hlstd{(sum_pdf[n])}
  \hlstd{\}}
  \hlkwd{return}\hlstd{(loglike)}
\hlstd{\}}

\hlstd{initialValues} \hlkwb{=} \hlkwa{function}\hlstd{(}\hlkwc{x}\hlstd{,} \hlkwc{K}\hlstd{,} \hlkwc{reps} \hlstd{=} \hlnum{100}\hlstd{)\{}
  \hlstd{mu} \hlkwb{=} \hlkwd{rnorm}\hlstd{(K,} \hlkwd{mean}\hlstd{(x),} \hlnum{5}\hlstd{)}
  \hlstd{sigma} \hlkwb{=} \hlkwd{sqrt}\hlstd{(}\hlkwd{rgamma}\hlstd{(K,} \hlnum{5}\hlstd{))}
  \hlstd{p} \hlkwb{=} \hlkwd{runif}\hlstd{(K)}
  \hlstd{p} \hlkwb{=} \hlstd{p}\hlopt{/}\hlkwd{sum}\hlstd{(p)}
  \hlstd{currentLogLik} \hlkwb{=} \hlkwd{loglik}\hlstd{(x, p, mu, sigma)}
  \hlkwa{for}\hlstd{(i} \hlkwa{in} \hlnum{1}\hlopt{:}\hlstd{reps)\{}
    \hlstd{mu_temp} \hlkwb{=} \hlkwd{rnorm}\hlstd{(K,} \hlkwd{mean}\hlstd{(x),} \hlnum{10}\hlstd{)}
    \hlstd{sigma_temp} \hlkwb{=} \hlkwd{sqrt}\hlstd{(}\hlkwd{rgamma}\hlstd{(}\hlnum{10}\hlstd{,} \hlnum{5}\hlstd{))}
    \hlstd{p_temp} \hlkwb{=} \hlkwd{runif}\hlstd{(K)}
    \hlstd{p_temp} \hlkwb{=} \hlstd{p_temp}\hlopt{/}\hlkwd{sum}\hlstd{(p_temp)}
    \hlstd{tempLogLik} \hlkwb{=} \hlkwd{loglik}\hlstd{(x, p_temp, mu_temp, sigma_temp)}
    \hlkwa{if}\hlstd{(tempLogLik} \hlopt{>} \hlstd{currentLogLik)\{}
      \hlstd{mu} \hlkwb{=} \hlstd{mu_temp}
      \hlstd{sigma} \hlkwb{=} \hlstd{sigma_temp}
      \hlstd{p} \hlkwb{=} \hlstd{p_temp}
      \hlstd{currentLogLik} \hlkwb{=} \hlstd{tempLogLik}
    \hlstd{\}}
  \hlstd{\}}
  \hlkwd{return}\hlstd{(}\hlkwd{list}\hlstd{(}\hlstr{"mu"} \hlstd{= mu,} \hlstr{"sigma"} \hlstd{= sigma,} \hlstr{"p"} \hlstd{= p))}
\hlstd{\}}

\hlcom{##################################################}
\hlcom{# EM algo}
\hlcom{##################################################}
\hlstd{EM} \hlkwb{=} \hlkwa{function}\hlstd{(}\hlkwc{x}\hlstd{,} \hlkwc{K}\hlstd{,} \hlkwc{tol} \hlstd{=} \hlnum{0.001}\hlstd{)\{}
  \hlstd{inits} \hlkwb{=} \hlkwd{initialValues}\hlstd{(x, K,} \hlnum{1000}\hlstd{)}
  \hlstd{mu} \hlkwb{=} \hlstd{inits}\hlopt{$}\hlstd{mu}
  \hlstd{sigma} \hlkwb{=} \hlstd{inits}\hlopt{$}\hlstd{sigma}
  \hlstd{prob} \hlkwb{=} \hlstd{inits}\hlopt{$}\hlstd{p}

  \hlstd{prevLoglik} \hlkwb{<-} \hlnum{0}
  \hlstd{loglikDiff}\hlkwb{<-} \hlnum{1}

  \hlcom{# while loop}
  \hlkwa{while}\hlstd{(loglikDiff} \hlopt{>} \hlstd{tol)\{}

    \hlstd{gamma} \hlkwb{<-} \hlkwd{gammaUpdate}\hlstd{(x, mu, sigma, prob)}
    \hlstd{mu} \hlkwb{<-} \hlkwd{muUpdate}\hlstd{(x, gamma)}
    \hlstd{sigma} \hlkwb{<-} \hlkwd{sigmaUpdate}\hlstd{(x, gamma, mu)}
    \hlstd{prob} \hlkwb{<-} \hlkwd{piUpdate}\hlstd{(gamma)}

    \hlstd{currentLogLik} \hlkwb{<-} \hlkwd{loglik}\hlstd{(x, prob, mu, sigma)}

    \hlstd{loglikDiff} \hlkwb{<-} \hlkwd{abs}\hlstd{(prevLoglik} \hlopt{-} \hlstd{currentLogLik)}

    \hlstd{prevLoglik} \hlkwb{<-} \hlstd{currentLogLik}

  \hlstd{\}}

  \hlkwd{return}\hlstd{(}\hlkwd{list}\hlstd{(}\hlstr{'loglik'} \hlstd{= currentLogLik,} \hlstr{'mu'} \hlstd{= mu,} \hlstr{'sigma'} \hlstd{= sigma,} \hlstr{'prob'} \hlstd{= prob))}
\hlstd{\}}

\hlkwd{set.seed}\hlstd{(}\hlnum{1996}\hlstd{)}
\hlstd{final_plot} \hlkwb{=} \hlkwd{matrix}\hlstd{(}\hlnum{0}\hlstd{,} \hlkwc{ncol} \hlstd{=} \hlnum{4}\hlstd{,} \hlkwc{nrow}\hlstd{=} \hlkwd{length}\hlstd{(galaxies))}
\hlstd{loglik_values} \hlkwb{=} \hlkwa{NULL}
\hlkwa{for}\hlstd{(k} \hlkwa{in} \hlnum{2}\hlopt{:}\hlnum{5}\hlstd{)\{}
  \hlstd{z} \hlkwb{=} \hlkwd{EM}\hlstd{(galaxies, k)}
  \hlstd{loglik_values[(k}\hlopt{-}\hlnum{1}\hlstd{)]} \hlkwb{=} \hlstd{z}\hlopt{$}\hlstd{loglik}
  \hlkwa{for}\hlstd{(i} \hlkwa{in} \hlnum{1}\hlopt{:}\hlstd{k)\{}
    \hlstd{final_plot[,(k}\hlopt{-}\hlnum{1}\hlstd{)]} \hlkwb{=} \hlstd{final_plot[,(k}\hlopt{-}\hlnum{1}\hlstd{)]} \hlopt{+} \hlstd{z}\hlopt{$}\hlstd{prob[i]} \hlopt{*} \hlkwd{dnorm}\hlstd{(galaxies, z}\hlopt{$}\hlstd{mu[i], z}\hlopt{$}\hlstd{sigma[i])}
  \hlstd{\}}
\hlstd{\}}

\hlstd{loglik_values} \hlkwb{<-} \hlkwd{as.data.frame}\hlstd{(loglik_values)} \hlopt
  \hlkwd{mutate}\hlstd{(}\hlstr{"K"} \hlstd{= (}\hlnum{2}\hlopt{:}\hlnum{5}\hlstd{))}

\hlstd{loglik_plot} \hlkwb{=} \hlkwd{ggplot}\hlstd{(}\hlkwc{data} \hlstd{= loglik_values)} \hlopt{+}
  \hlkwd{geom_line}\hlstd{(}\hlkwd{aes}\hlstd{(}\hlkwc{x}\hlstd{=K,} \hlkwc{y}\hlstd{=loglik_values),} \hlkwc{color} \hlstd{=} \hlstr{"blue"}\hlstd{)} \hlopt{+}
  \hlkwd{labs}\hlstd{(}\hlkwc{x}\hlstd{=}\hlstr{"Number of components"}\hlstd{,}\hlkwc{y}\hlstd{=} \hlstr{"Log likelihood"}\hlstd{)}
\hlstd{loglik_plot}
\end{alltt}
\end{kframe}
\includegraphics[width=\maxwidth]{figure/unnamed-chunk-7-2} 
\begin{kframe}\begin{alltt}
\hlstd{final_plot} \hlkwb{=} \hlkwd{as.data.frame}\hlstd{(final_plot)}
\hlkwd{colnames}\hlstd{(final_plot)} \hlkwb{=} \hlkwd{c}\hlstd{(}\hlstr{"K = 2"}\hlstd{,} \hlstr{"K = 3"}\hlstd{,} \hlstr{"K = 4"}\hlstd{,} \hlstr{"K = 5"}\hlstd{)}
\hlstd{final_plot} \hlkwb{=} \hlkwd{cbind}\hlstd{(final_plot, galaxies)}

\hlkwd{ggplot}\hlstd{(}\hlkwc{data} \hlstd{= final_plot,} \hlkwd{aes}\hlstd{(}\hlkwc{x} \hlstd{= galaxies))} \hlopt{+}
  \hlkwd{geom_line}\hlstd{(}\hlkwd{aes}\hlstd{(}\hlkwc{y} \hlstd{= `K = 2`,} \hlkwc{color} \hlstd{=} \hlstr{"K = 2"}\hlstd{),} \hlkwc{size} \hlstd{=} \hlnum{1}\hlstd{)} \hlopt{+}
  \hlkwd{geom_line}\hlstd{(}\hlkwd{aes}\hlstd{(}\hlkwc{y} \hlstd{= `K = 3`,} \hlkwc{color} \hlstd{=} \hlstr{"K = 3"}\hlstd{),} \hlkwc{size} \hlstd{=} \hlnum{1}\hlstd{)} \hlopt{+}
  \hlkwd{geom_line}\hlstd{(}\hlkwd{aes}\hlstd{(}\hlkwc{y} \hlstd{= `K = 4`,} \hlkwc{color} \hlstd{=} \hlstr{"K = 4"}\hlstd{),} \hlkwc{size} \hlstd{=} \hlnum{1}\hlstd{)} \hlopt{+}
  \hlkwd{geom_line}\hlstd{(}\hlkwd{aes}\hlstd{(}\hlkwc{y} \hlstd{= `K = 5`,} \hlkwc{color} \hlstd{=} \hlstr{"K = 5"}\hlstd{),} \hlkwc{size} \hlstd{=} \hlnum{1}\hlstd{)} \hlopt{+}
  \hlkwd{geom_density}\hlstd{(}\hlkwd{aes}\hlstd{(}\hlkwc{fill} \hlstd{=} \hlstr{"Density plot"}\hlstd{),} \hlkwc{color} \hlstd{=} \hlstr{"pink"}\hlstd{,} \hlkwc{alpha} \hlstd{=} \hlnum{0.2}\hlstd{,} \hlkwc{size} \hlstd{=} \hlnum{0}\hlstd{)} \hlopt{+}
  \hlkwd{labs}\hlstd{(}\hlkwc{x} \hlstd{=} \hlstr{"km"}\hlstd{,} \hlkwc{y} \hlstd{=} \hlstr{"Density"}\hlstd{,}\hlkwc{title}\hlstd{=}\hlstr{"Plot of different values of K and the density plot of galaxies"}\hlstd{)} \hlopt{+}
  \hlkwd{theme}\hlstd{(}\hlkwc{legend.title} \hlstd{=} \hlkwd{element_blank}\hlstd{(),}\hlkwc{legend.position} \hlstd{=} \hlkwd{c}\hlstd{(}\hlnum{.95}\hlstd{,} \hlnum{.95}\hlstd{),}
        \hlkwc{legend.justification} \hlstd{=} \hlkwd{c}\hlstd{(}\hlstr{"right"}\hlstd{,} \hlstr{"top"}\hlstd{),}
        \hlkwc{legend.box.just} \hlstd{=} \hlstr{"right"}\hlstd{,}
        \hlkwc{legend.margin} \hlstd{=} \hlkwd{margin}\hlstd{(}\hlnum{6}\hlstd{,} \hlnum{6}\hlstd{,} \hlnum{6}\hlstd{,} \hlnum{6}\hlstd{))}
\end{alltt}
\end{kframe}
\includegraphics[width=\maxwidth]{figure/unnamed-chunk-7-3} 

\end{knitrout}

\end{document}
